\documentclass{article}
% common math packages
\usepackage{amsmath}
\usepackage{amsthm}
\usepackage{amssymb}
% this is for figures
\usepackage{graphicx}
\usepackage{float}
% this is for adjusting margins
\usepackage{geometry}
\geometry{
	a4paper,
	left=21mm,
	right=21mm,
	top=33mm,
	bottom=38mm
}
% this is for managing theorems, lemmas, etc
\newtheorem{definition}{Definition}[section]
\newtheorem{theorem}[definition]{Theorem}
\newtheorem{lemma}[definition]{Lemma}
\newtheorem{corollary}[definition]{Corollary}
\newtheorem{remark}[definition]{Remark}

% this is for creating lists
\usepackage{enumerate}

% this is for links
\usepackage{hyperref}
% this is the title
\title{Something}
\date{}
\author{Something}

% this part is known as the TEXT
\begin{document}
	%\maketitle
	\section{Introduction}
		\subsection{Lines of Text}
			Something This is a text. However,
			entering
			a 
			new line
			does not create a new line. 
			This will create a new line: [Something]
			This will also create a new line: [Something]
			This creates a new line 5.75 millimetres below:	[Something]
			This is a new paragraph: 
			[Something]
		\subsection{Text Formatting}
			Some formatting techniques:\\[3mm]
			Something{this is a bold-faced text}\\[3mm]
			Something{this is an italicised text}\\[3mm]
			Something{this is an underlined text}\\[3mm]
			%Something Interesting %
			%Something in the center%
		\subsection{Some Special Characters}
			Some special characters are:
			\begin{center}
				Somethings
			\end{center}
			Here are some ways of writing accents:			
			\begin{center}
				Somethings
			\end{center}
			[Something]This is a bad quote[Something]...\\[3mm]
			\emph{Remark.} Never use [Something]\\[3mm]
			[Something]This is a good quote[Something]\\
			and [Something]This is also a good quote[Something]
	% this is the coolest section by far
	% And we want to give it a new page
	\section{Mathematics}
		In this section, we will discuss essential features of the mathematics environment.
		\subsection{Basic Mathematics}
			The most basic mathematical environment is this [Sth]x + y - z = 1[Sth].\\
			Compare it with x + y - z = 1.\\
			Here's a list of basic math features:
			\begin{enumerate}
				\item Greek Symbols: Something.
				\item Common Symbols: Something.
				\item More Common Symbols: Something.
				\item Elementary Functions: Something, remember not to write [Something] etc.
				\item Some more: Something.
				\item Even more: \href{http://detexify.kirelabs.org/classify.html}{Detexify} %or \url{http://detexify.kirelabs.org/classify.html}
			\end{enumerate}
			Let's talk about fractions, subscripts and superscripts...
			\begin{enumerate}
				\item Fractions: Something
				\item Subscripts: Something
				\item Superscripts: Something
				\item Combinations: Something as Something... 				\item Always remember, subscripts first followed by superscripts, e.g. Somethings
			\end{enumerate}
		\subsection{Mathematical Fonts}
			Here are some useful fonts:
			\begin{enumerate}
				\item Default: Something
				\item Other fonts:
				\begin{enumerate}[(a)]
					\item Blackboard Bold: Something
					\item Bold-faced: Something
					\item Caligraphy: Something
					\item Fraktur: Something
				\end{enumerate}
			\end{enumerate}
		\subsection{Equations}
			% itemize: creates an unnumbered list.
			\begin{itemize}
				\item An example of an numbered equation is
				% %
				\item An example of an unnumbered equation is
				% %
				\item This is how we split equations:
				% %
			\end{itemize}
		\subsection{Integrals, Limits, Summation}
			\begin{itemize}
				\item Integral: Something.
				\item Limit: Something.
				\item Summation: Something.
				\item Partial Derivative: Something
			\end{itemize}
			
		\subsection{Matrices, Vectors}
			A Matrix:
			% pmatrix %
			% bmatrix %
			A Vector:
			% normal way %
			% lazy way %
			
		% this is mainly for math majors...
		\subsection{Theorems, Lemma, Definitions, Corollaries}
			%definition
				Let $(x_{n})_{n=1}^{\infty}$ be a real-valued sequence. A sequence is said to converge to a limit $x \in \mathbb{R}$ if for any given $\varepsilon$ ,there exists $N \in\mathbb{N}$ such that $n \geq N$ implies $\|x_{n}-x\|<\varepsilon$. 
			
			%refer to label
			Based on Definition[Something] we have the following lemma.
			%lemma
			If $x_{n} \to x$, then $|x_{n}| \to |x|$.
			We can prove this lemma by:
			
			%proof
			
			% Remark %
			
			%theorem
			Let $X_{1}, X_{2}, X_{3}, \dots$ be a sequence of IID random variables, with finite mean $\mu$ and variance $\sigma^{2}$. Then
				\begin{equation}
					\frac{\bar{X}-\mu}{\sqrt{\sigma^{2}/n}}
				\end{equation}
				can be approximated by the standard normal distribution.


	\newpage
	\section{Others}
		\subsection{Tables}
			
		\subsection{Figures}
			
			
	% basic bibliography
	\begin{thebibliography}{100}
		\bibitem{latex-course} Jiaming Song (2018), \emph{``A First Course in \LaTeX{}''}. NUS Mathematics Society.
	\end{thebibliography}
	
\end{document}
