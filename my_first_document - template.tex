% percentage sign is a comment, everything after '%' will not be compiled or shown in the pdf
% F1 is a shortcut key for compiling/quick building a pdf preview
% Each time you compiled this file, the pdf generated is overwitten.

\documentclass{article} %Indicating what type of documents you are writing. 'article' for a normal pdf, 'beamer' for ppt slides, 'exam' for an exam template

% importing useful packages

% common math packages
\usepackage{amsmath}
\usepackage{amsthm}
\usepackage{amssymb}

% this is for figures
\usepackage{graphicx}
\usepackage{float}

% this is for adjusting margins
\usepackage{geometry}
\geometry{
	a4paper,
	left=21mm,
	right=21mm,
	top=21mm,
	bottom=21mm
}

% this is for managing theorems, lemmas, etc
\newtheorem{definition}{Definition}[section]
\newtheorem{theorem}[definition]{Theorem}
\newtheorem{lemma}[definition]{Lemma}
\newtheorem{corollary}[definition]{Corollary}
\newtheorem{remark}[definition]{Remark}

% this is for creating lists
\usepackage{enumerate}

% this is for links
\usepackage{hyperref}

% this is the title
\title{Latex Workshop}
\date{Jan 24,2018} 

% if you want the compiler to auto generate today's date, you don't need this line
% if you do not want the date, just use: \date{}
\author{Jiaming}


% all the parts above are called preamble, the preamble is mainly for you to import packages, define new environment or packages, and those will not appear in the pdf generated.

% you don't have to memorize all these packages, can just copy over whenever you want to write a new file. For more packages: just google

% this part is known as the TEXT
% everything to be shown in the pdf, must be written within the \begin{documment} EVERYTHING \end{document}
% note that whenever there is a '\begin{sth}', there has to be a '\end{sth}'. 
\begin{document}
	\maketitle % this command will display whatever you defined for the titles in the preamble section. \title{}, \date{}, \author{•}
	\tableofcontents % display the table of contents, must push other stuff to the next page for it to be displayed
	\newpage % contents after this will be displayed in a new page
	\section{Introduction}
		\subsection{Lines of Text}
		% \hspace{4mm} creates a horizontal space of 4mm
			\hspace{4mm} Something This is a text. However,
			entering
			a 
			new line
			does not create a new line. 
			This will create a new line: \\
			This will also create a new line: \newline
			This creates a new line 5.75 millimetres below:	\\[5.75mm]
			This is a new paragraph: 
			\par This is a new paragraph. This is a new paragraph. This is a new paragraph. This is a new paragraph. This is a new paragraph. This is a new paragraph
		\subsection{Text Formatting}
			Some formatting techniques:\\[3mm]
			\textbf{this is a bold-faced text}\\[3mm]
			\emph{this is an italicised text}\\[3mm]
			\underline{this is an underlined text}\\[3mm]
			%Something Interesting %
			\textbf{\emph{\underline{interesting}}}
			%Something in the center%
			\begin{center}
				This is centralised
			\end{center}
			% don't use '\center{}' !!
		\subsection{Some Special Characters}
			Some special characters are:
			% These special characters have special usage in Latex, i.e '%' indicates a comment, '$' indicates a math environment, thus if we want to type them as plaintext, we have to put a '\' in front of them.
			\begin{center}
				\%, \$, \&, \{, \}
			\end{center}
			More special characters:
			\begin{center}
			$\backslash$, \^{} % backslash itself and ^
			\end{center}
			 
			Here are some ways of writing accents:			
			\begin{center}
				L'H\^{o}pital, March\'{e}, H\"{o}lder's Inequality
			\end{center}
			"This is a bad quote"...\\[3mm]
			\emph{Remark.} Never use "\\[3mm]
			`This is a good quote'\\ % ` is located to the left 1, and under '~'
			and ``This is also a good quote''
	
	\newpage
	\section{Mathematics}
		In this section, we will discuss essential features of the mathematics environment.
		\subsection{Basic Mathematics}
			The most basic mathematical environment is this $x + y - z = 1$.Compare it with x + y - z = 1.\\
			`\$ blah blah \$' gives inline math environment. i.e. the math formula is displayed in the same line as other words. However, using double `\$\$ blah blah \$\$' will display the formula in a line itself. For example:\\
			What is the different between $ 1+1=2$ and $$1+1=2$$ equation?\\
			
			Here's a list of basic math features:
			% \begin{enumerate} ... \end{enumerate} creates a numbered list environment. Inside the environment, put '\item' in front of each point.
			\begin{enumerate}
				\item Greek Symbols: $\alpha,\beta,\sigma,\pi,\epsilon,\varepsilon, \Sigma,\Pi$ etc.
				\item Common Symbols: $\{1,2,3\} \subseteq \{1,2,3,4\}, 1 \in \{1,2\},\forall, \exists, \leq, \geq$.
				\item More Common Symbols: $\infty \not \in \{1,2\}$.
				\item Elementary Functions: $\cos(x),\sin(x),\exp(x),\min, \max, \ln(x),\log(x)$, remember not to write $sin(x)$ etc.
				\item Some more: $f:A\to B, B \implies C, C \equiv D, D\Leftrightarrow E, E \circ F$.
				\item Even more: Use this website to find the name of the symbol!: \href{http://detexify.kirelabs.org/classify.html}{Detexify}  % only word 'Detexify' will be displayed, but it links to the url on click
				% another way to attach just the link: \url{http://detexify.kirelabs.org/classify.html}
			\end{enumerate}
			Let's talk about fractions, subscripts and superscripts...
			\begin{enumerate}
				\item Fractions: $\frac{1}{2}, \frac{1}{1+\frac{1}{n}}$
				\item Subscripts: $x_{1},x_{2},\dots,x_{2018}$
				\item Superscripts: $x^{1},x^{2},\dots,x^{2018}$
				\item Combinations: $(1+\frac{1}{n})^{n} \to e$ as $n \to \infty$... 				
				\item Always remember, subscripts first followed by superscripts, e.g. $\alpha_{x}^{\epsilon}, x_{1_{2}}^{3^{4}}$
			\end{enumerate}
		\subsection{Mathematical Fonts}
			Here are some useful fonts:
			\begin{enumerate}
				\item Default: $a,b,A,B$
				\item Other fonts:
				\begin{enumerate}[(a)] % to number the points using a,b,c,d instead of just 1,2,3,4
				% can also use \begin{enumerate}[(i)] to number the points using i, ii, iii
					\item Blackboard Bold: $\mathbb{Q}\subset\mathbb{R}$
					\item Bold-faced: $\mathbf{P}(X=k)=0.4$
					\item Caligraphy: $A \in \mathcal{A}\cup\mathcal{B}$ 
					\item Fraktur: $\mathfrak{ABCDEFG}$
				\end{enumerate}
			\end{enumerate}
		\subsection{Equations}
			% itemize: creates an unnumbered list.
			\begin{itemize}
				\item An example of an numbered equation is
				\begin{equation}
				 e^{i \pi}+1 =0
				\end{equation}
				% note that the area between \begin{equation} and \end{equation} is math environment, you do not have to use $ inside.
				\item An example of an unnumbered equation is
				\begin{equation*}
				e^{i \pi}+2 \not=0
				\end{equation*}
				\item This is how we split equations:
				\begin{equation}
					\begin{split}
					(1+2)+3 &=6\\
							&= 3+ 3\\
							&= 6.0
					\end{split}
				\end{equation}
			\end{itemize}
		\subsection{Integrals, Limits, Summation}
			\begin{itemize}
				\item Integral: $\int_{0}^{1} f(x) \, dx$. \emph{Note}: `$\backslash$,' creates an extra space between items in math environment. By default, space is not create by just typing ` '. i.e. Compare $A B$, $AB$ and $A \, B$
				\item Limit: $\lim_{n\to\infty}(1+\frac{1}{n})^{n} = e$.
				\item Summation: $\sum\limits_{n=0}^{\infty} \frac{1}{n!} = e$.
				\item Partial Derivative: $\frac{\partial f}{\partial x}$
			\end{itemize}
			
		\subsection{Matrices, Vectors}
			A Matrix:
			% pmatrix % (round brackets)
			\begin{equation}
			% matrix must be placed inside a math environment, except 'equation', you can also place it inside $..$ or $$..$$
			\begin{pmatrix}
			1 & 2 & 3\\
			4 & 5 & 6\\
			7 & 8 & 9
			\end{pmatrix}
			\end{equation}
			% bmatrix % (square brackets)
			$$\begin{bmatrix}
			 1 & \dots & \cdots\\
			4 & \ddots & \vdots\\
			7 & 8 & 9
			\end{bmatrix}$$
			A Vector:
			% normal way %
			\begin{equation}
			\vec{v} =
			\begin{bmatrix}
			x_{1}\\
			x_{2}\\
			x_{3}
			\end{bmatrix}
			\end{equation}
			% lazy way %
			$\vec{v} = [1,2,3]^{T}$
			
		% this is mainly for math majors...
		\subsection{Theorems, Lemma, Definitions, Corollaries}
			\begin{definition}\label{convergence} % label is used to be refered later
				Let $(x_{n})_{n=1}^{\infty}$ be a real-valued sequence. A sequence is said to converge to a limit $x \in \mathbb{R}$ if for any given $\varepsilon$ ,there exists $N \in\mathbb{N}$ such that $n \geq N$ implies $\|x_{n}-x\|<\varepsilon$. 
			\end{definition}
			Based on Definition~\ref{convergence} we have the following lemma. % auto refered to convergence label
			\begin{lemma}\label{simple lemma}
			If $x_{n} \to x$, then $|x_{n}| \to |x|$.
			\end{lemma}
			We can prove this lemma by:
			\begin{proof}
			Left as an exercise.
			\end{proof}
			% Remark %
			\begin{remark}
			The converse of Lemma~\ref{simple lemma} is not true in general
			\end{remark}
			%theorem
			\begin{theorem}
			Let $X_{1}, X_{2}, X_{3}, \dots$ be a sequence of IID random variables, with finite mean $\mu$ and variance $\sigma^{2}$. Then
				\begin{equation}
					\frac{\bar{X}-\mu}{\sqrt{\sigma^{2}/n}} 
				\end{equation}
				can be approximated by the standard normal distribution.
			\end{theorem}

	\newpage
	\section{Others}
		\subsection{Tables}
		Refer to \cite{latex-course} % Refer to the reference (at the end) item with label ‘latex-course’
		\begin{center} % can remove the center environment if you dont want to put the table in the center
			\begin{tabular}{|c|c|c|} % {|c|c|c|} how many c indicates how many columns. Here we create a table of 3 columns
			\hline  %draw a horizontal line
			Name & Age & Favoured Fruit\\
			\hline
			Jiaming & 15 & Apple\\
			Joel & 2 & Whatever\\
			\hline
			\end{tabular}
		\end{center}
		\subsection{Figures}
		\begin{figure}[H] % [H] means to place the picture here, if without [H], the picture will be displayed at the top of the page
		\centering 
			\includegraphics[width = 0.5\textwidth]{what.jpg} % include an image, with image width = half of the width of text. what.jpg if the name of the image you want to include. 
			\caption{what???}
		\end{figure}
			
			
	% basic bibliography /reference
	\begin{thebibliography}{100} % {100} means at most you have 100 item, \bibitem works the same way as the \item in enumerate or itemize
		\bibitem{latex-course} Jiaming Song (2018), \emph{``A First Course in \LaTeX{}''}. NUS Mathematics Society.
		% latex-course is the label, so you can refer to this reference in other part of your report. See the line under 'Table' section.
		\bibitem{useful links} Sharelatex, \url{https://www.sharelatex.com/learn/}
	\end{thebibliography}
	
\end{document}
